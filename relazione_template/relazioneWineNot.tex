%%%%%%%%%%%%%%
%  COSTANTI  %
%%%%%%%%%%%%%%
%COMANDI DA UTILIZZARE PER TUTTI I DOCUMENTI
% In questa prima parte vanno definite le 'costanti' utilizzate da due o più documenti.

%COMANDI TABELLE
\newcommand{\rowcolorhead}{\rowcolor[HTML]{00ACC1}} %intestazione 
\newcommand{\rowcolorlight}{\rowcolor[HTML]{E0f7fa}} %righe chiare/dispari
\newcommand{\rowcolordark}{\rowcolor[HTML]{80DEEA}} %righe scure/pari
\newcommand{\colorhead}{\color[HTML]{FFFFFF}} %testo intestazione
\newcommand{\colorbody}{\color[HTML]{000000}} %testo righe


%%%%%%%%%%%%%%
%  FUNZIONI  %
%%%%%%%%%%%%%%

% Serve a dare la giusta formattazione al codice inline
\newcommand{\code}[1]{\flextt{\texttt{#1}}}

% Genera automaticamente la pagina di copertina
\newcommand{\makeFrontPage}{
  % Declare new geometry for the title page only.
  \newgeometry{top=4cm}
  
  \begin{titlepage}
  \begin{center}

\begin{center}
  \centerline{\includegraphics[scale=0.24]{StyleLatex/logo_home.png}}
\end{center}
  
  \vspace{1cm}

  \begin{Huge}
  \textbf{\DocTitolo{}} \\
  \end{Huge}

  \vspace{9pt}  
  
  \begin{tabular}{l|l}
  	\textbf{Nome Gruppo} & I Tesori di Squitty \\
  	\textbf{Componenti} & Alessio Gobbo 1125947\\
  	& Simone Ballarin 1122286\\
  	& Riccardo Dario\\
  	& Gerta Lleshi\\
  	\textbf{Dati login utente} & username: kevinsilvstr \\
  	& password: Kevin940 \\
  	\textbf{Dati login admin} & username: admin \\
  	& password: Admin420 \\
  	\textbf{Username consegna} & ksilvest\\ 
  	\hline
  \end{tabular}

  \vspace{30pt}
  
\begin{center}
	\centerline{Indirizzo sito: 
	http://tecweb2016.studenti.math.unipd.it/ksilvest/WineNot/}
\end{center}

  \end{center}
  \end{titlepage}
  
  % Ends the declared geometry for the titlepage
  \restoregeometry
}
\input{StyleLatex/layout.tex}
%%%%%%%%%%%%%%
%  COSTANTI  %
%%%%%%%%%%%%%%

% In questa prima parte vanno definite le 'costanti' utilizzate soltanto da questo documento.
% Devono iniziare con una lettera maiuscola per distinguersi dalle funzioni.

\newcommand{\DocTitolo}{Relazione progetto TecWeb}

\newcommand{\DocRedazione}{I Tesori di Squitty}

\newcommand{\DocAnno}{Anno 2017/2018}

\title{\textbf{Relazione progetto TecWeb}}
\author{Roditori Pungenti}

\date{02 Luglio 2018}

\begin{document}

%\maketitle

\makeFrontPage

\tableofcontents

\newpage

\section{Abstract}

L’azienda dolciaria “I TESORI DI SQUITTY” possiede un negozio situato nel centro cittadino e un laboratorio industriale in periferia.\\
Lo scopo del progetto è quello di progettare e sviluppare un sito web che funga da vetrina e di erogare un servizio automatizzato di ordinazioni.\\

Qui di seguito sono elencati tutti i livelli di accessibilità possibili per il sito:
\begin{itemize}
	\item \textbf{Unauthenticated user}: identifica qualsiasi utente;
	\item \textbf{Authenticated user}: identifica un utente con account "al minuto"/"all'ingrosso"/"servizio" che ha effettuato l'accesso al portale web;
	\item \textbf{Authenticated administrator}: identifica un utente con account impiegato che ha effettuato l'accesso al portale web.
\end{itemize}

L'\textbf{unauthenticated user} può accedere alle pagine \textit{Home}, \textit{Per la tua casa}, \textit{Per il tuo ristorante}, \textit{Catering ed Eventi}, \textit{Accedi}, \textit{Registrati}:
\begin{itemize}
	\item \textbf{Home}: in questa pagina può trovare la storia dell'azienda dolciaria, nonchè numerose informazioni sia sul punto vendita che sullo stabilimento produttivo.

	\item \textbf{Per la tua casa}: in questa pagina può visualizzare l'immagine, gli ingredienti e la descrizione di tutti i prodotti disponibili per il commercio al dettaglio.

	\item \textbf{Per il tuo ristorante}: in questa pagina può consultare il catalogo,con relativa descrizione, dei prodotti disponibili per il commercio all'ingrosso.

	\item \textbf{Catering ed Eventi}: in questa pagina può visualizzare i servizi erogati dall'azienda.

	\item \textbf{Registrati}: in questa pagina l'utente non autenticato può effettuare la registrazione, scegliendo fra le tipologie utente: al minuto, all'ingrosso, servizio.

	\item \textbf{Accedi}: in questa pagina l'utente può autenticarsi, previo inserimento delle credenziali.\\
	Egli verrà reindirizzato alla propria area privata.
\end{itemize}

L'\textbf{authenticated user} può accedere alle sua area personale, suddivisa in \textit{Prenotazione}, \textit{Storia degli ordini}, \textit{Prodotti}.

\begin{itemize}
	\item \textbf{Prenotazione}: in questa pagina può prenotare un prodotto/servizio previo inserimento dei dati necessari, quali: quantità, data e ora di ritiro ecc...

	\item \textbf{Storia degli ordini}: in questa pagina l'utente autenticato può consultare lo storico delle prenotazioni effettuate, visualizzarne i dettagli e annallarle.

	\item \textbf{Prodotti}: in questa pagina può visualizzare il catalogo dei prodotti ordinabili compatibili con la propria tipologia di account.
\end{itemize}

Gli \textbf{authenticated administrator}, con account di tipo "Impiegato", hanno accesso alle sezioni: \textit{Ordini},\textit{Utenti}, \textit{Prodotti}.

\begin{itemize}
	\item \textbf{Ordini:} in questa pagina l'impiegato può consultare le ordinazioni divise per tipologia, visualizzarne i dettagli, cambiarne lo stato (in lavorazione ecc..) ed eliminarle.

	\item \textbf{Utenti:} in questa pagina l'impiegato visualizza l'elenco di tutti gli utenti registrati al portale web, compresi i dati anagrafici e le credenziali d'accesso e procedere alla loro eliminazione.

	\item \textbf{Prodotti:} in questa pagina l'impiegato può visualizzare i dettagli di tutti i prodotti e i servizi offerti da I Tesori di Squitty ed eliminarli.
\end{itemize}

\subsection{Analisi}

\subsubsection{Utenti destinatari}

Si sono d'apprima individuati gli utenti destinatari, tramite i seguenti step:

\begin{itemize}
	\item \textbf{Individuazione delle categorie di utenti}

		\begin{figure}[h!]
			\includegraphics[width=1\linewidth]{StyleLatex/img/catUt.png}
		\end{figure}

	\item \textbf{Definizione di gruppo omogeneo di utenti}

	\begin{figure}[h!]
		\includegraphics[width=1\linewidth]{StyleLatex/img/GrOm.png}
	\end{figure}

	\newpage

	\item \textbf{Individuazione degli obiettivi per classe d'utenza}

	\begin{figure}[h!]
		\includegraphics[width=1\linewidth]{StyleLatex/img/obiettivi.png}
	\end{figure}

\end{itemize}


\subsubsection{Base informativa}

Si è poi stilata una base informativa da cui attingere, andando a definire:

\begin{itemize}
	\item logo dell'azienda 
	\item storia dell'azienda 
	\item descrizione dello stabilimento 
	\item descrizione del negozio
	\item i prodotti
	\item i servizi
	\item contatti
\end{itemize} 

\subsubsection{Progettazione}

La progettazione è stata effettuata perseguendo il paradigma della programmazione ad oggetti.\\
L'architettura si suddivide in 3 aree principali: backend, communication protocol, frontend.

Per quanto concerne il backend, esso implementa la logica di business e si interfaccia con il database, garantendo la persistenza.

\begin{figure}[h!]
	\includegraphics[width=1\linewidth]{StyleLatex/PHP_backend.jpg}
\end{figure}

\newpage

Questo "package" modella gli oggetti che rappresentano lo stato delle entità contenute nel database e dei dati inseriti tramite il layer di presentazione.

\begin{figure}[h!]
	\includegraphics[width=1\linewidth]{StyleLatex/PHP_comunication_protocol.jpg}
\end{figure}



\begin{figure}[h!]
	\includegraphics[width=1\linewidth]{StyleLatex/PHP_frontend.jpg}
\end{figure}

\subsection{Validazione}

Gli strumenti utilizzati per la validazione sono: \textbf{W3C Markup Validation Service} (per validazione HTML), \textbf{W3C CSS Validator} (per validazione CSS), \textbf{Total Validator} (per validazione accessibilità).\\
Essendo prive di errori e warning, tutte le pagine sono state validate con successo.

\begin{figure}[h!]
	\includegraphics[width=0.2\linewidth]{StyleLatex/badge-html5.png}
\end{figure}

\section{Struttura del progetto}

Per gestire la complessità del progetto e mantenere la separazione fra contenuto,stile e comportamento i file che compongono il portale sono stati raggruppati e suddivisi in 4 cartelle diverse:

\begin{itemize}
	\item \textbf{css:} questa cartella contiene tutti i file riguardanti il css, opportunamente divisi in tre file, per desktop, mobile e stampa;
	\item \textbf{php:} questa cartella contiene i file php del sito web;
	\item \textbf{img:} questa cartella contiene tutte le immagini, i loghi e le icone del sito web;
	\item \textbf{js:}  questa pagina contiene il codice javascript del progetto, ovvero la sola gestione dei coockies.
\end{itemize}

\newpage

\section{Gestione dei dati}

Il sito web si appoggia su una base di dati di tipo MySQL. Tutte le interazioni vengono lanciate da codice PHP tramite la libreria mysqli.\\
La struttura del database è la seguente:

\begin{figure}[h!]
	\includegraphics[width=1\linewidth]{StyleLatex/ERTecWeb.jpeg}
\end{figure}

\section{Accessibilità}

Sono state adottate le seguenti misure in modo da perseguire l'accessibilità:
\begin{itemize}
	\item Link all’interno di un'access bar per andare direttamente al contenuto;
	\item Link salta menù in ogni menù;
	\item Tab index per poter accedere in ordine sequenzialmente corretto ai contenuti;
	\item Freccia “Torna su” fissata a fondo pagina;
	\item Presenti i meta tag per facilitare l’acesso ai motori di ricerca e fornire in modo sintetico le info sul portale web;
	\item Tutte le pagine sono in italiano, le poche parole in inglese sono state inserite in uno <span></span> con attributo lang=”eng”.
	\item L'utilizzo limitato di codice Javascript persegue la non intrusività dello scripting.
	\item Link alla pagina corrente evidenziato e non cliccabile nella barra di navigazione;
	\item Il sito web inoltre implementa un layout responsive, nel completo rispetto dei vigenti standard di accessibilità.
	\item Link cliccabili in tutta l’area del contenitore nel quale sono contenuti e non solo nel link stesso;
	\item Sitemap gerarchica di tutte le pagine;
	\item Percorso seguito per raggiungere la pagina corrente;
	\item Link di navigazione interna alle pagine;
	\item Link già visitati di colorazione diversa rispetto a quelli non ancora visitati;
	\item Immagini chiaramente separate dal testo;
	\item Utilizzo del tool HTML5 shiv: esso permette alle versioni di IE < 9 di riconoscere i tag html5 e permetterne lo styling.
	\item La presenza di colori a contrasto elevato facilita la lettura anche a 	persone con capacità visive limitate.
	
\end{itemize}

Relativamente a quest'ultimo punto sono stati effettuati i dovuti test tramite color contrast checker.\\
Qui di seguito riportiamo gli schemi colori presi in esame:\\

\begin{tabular}{l|l}
	\hline
	 \textbf{Oggetto} & \textbf{Contrast ratio}  \\
	\hline
	  Contenuto pagine & 20.76:1  \\
	\hline
	Menù/briciole & 7.08:1 \\
	\hline
	Menù(voci inattive) & 7.00:1 \\
	\hline
	Link non visitati & 8.49:1 \\
	\hline
	Link visitati & 10.88:1 \\
	\hline
	Bottoni & 15.46:1 \\
\end{tabular}\\
Per ognuno schema, l'esito del test è stato il seguente:\\
\begin{tabular}{l|l}
	\hline
	 \textbf{WCAG AA} & Pass  \\
	\hline
	 \textbf{WCAG AAA} & Pass  \\
	\hline
\end{tabular}


\section{Compatibilità browser}

Il presente sito web è stato testato sui principali browser. Dai test effettuati, non sono emerse particolari problematiche che possano influire negativamente sulla navigazione o sull'usabilità del sito.
Di seguito si riporta un breve riepilogo delle principali piattaforme testate:
	\begin{enumerate}
		\item \textbf{Chrome}
		\item \textbf{Edge}
		\item \textbf{Firefox}
	\end{enumerate}

\section{Organizzazione del gruppo di lavoro}

La seguente tabella indica quali sono state le principali mansioni di ogni componente del gruppo. Ogni componente,tuttavia , ha contribuito almeno in parte ad ogni fase di sviluppo del progetto.

\subsection{Alessio Gobbo}

\begin{itemize}
	\item HTML
	\item PHP 
	\item CSS Desktop, Mobile
	\item Stesura relazione
\end{itemize}

\subsection{Simone Ballarin}

\begin{itemize}
	\item PHP 
	\item CSS Desktop, Mobile
	\item HTML
	\item Validazione
\end{itemize}

\subsection{Gerta Lleshi}

\begin{itemize}
	\item PHP
	\item Database
	\item Javascript
	\item HTML
\end{itemize}

\subsection{Riccardo Dario}

\begin{itemize}
	\item CSS Stampa
	\item Accessibilità
	\item Stesura relazione
\end{itemize}

\end{document}