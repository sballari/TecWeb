%%%%%%%%%%%%%%
%  COSTANTI  %
%%%%%%%%%%%%%%
%COMANDI DA UTILIZZARE PER TUTTI I DOCUMENTI
% In questa prima parte vanno definite le 'costanti' utilizzate da due o più documenti.

%COMANDI TABELLE
\newcommand{\rowcolorhead}{\rowcolor[HTML]{00ACC1}} %intestazione 
\newcommand{\rowcolorlight}{\rowcolor[HTML]{E0f7fa}} %righe chiare/dispari
\newcommand{\rowcolordark}{\rowcolor[HTML]{80DEEA}} %righe scure/pari
\newcommand{\colorhead}{\color[HTML]{FFFFFF}} %testo intestazione
\newcommand{\colorbody}{\color[HTML]{000000}} %testo righe


%%%%%%%%%%%%%%
%  FUNZIONI  %
%%%%%%%%%%%%%%

% Serve a dare la giusta formattazione al codice inline
\newcommand{\code}[1]{\flextt{\texttt{#1}}}

% Genera automaticamente la pagina di copertina
\newcommand{\makeFrontPage}{
  % Declare new geometry for the title page only.
  \newgeometry{top=4cm}
  
  \begin{titlepage}
  \begin{center}

\begin{center}
  \centerline{\includegraphics[scale=1.5]{StyleLatex/logo_home.png}}
\end{center}
  
  \vspace{1cm}

  \begin{Huge}
  \textbf{\DocTitolo{}} \\
  \end{Huge}

  \vspace{9pt}  
  
  \begin{tabular}{l|l}
  	\textbf{Nome Gruppo} & I Tesori di Squitty \\
  	\textbf{Componenti} & Alessio Gobbo 1125947\\
  	& Simone Ballarin 1122286\\
  	& Riccardo Dario 1123773\\
  	& Gerta Lleshi 1138601\\
  	\textbf{Dati login utente} & username:  \\
  	& password:  \\
  	\textbf{Dati login admin} & username:  \\
  	& password:  \\
  	\textbf{Username consegna} & sballari\\ 
  	\hline
  \end{tabular}

  \vspace{30pt}
  
\begin{center}
	\centerline{Indirizzo sito: 
	http://tecweb2016.studenti.math.unipd.it/sballari/TecWeb/Src/php/presentation/home.php}
\end{center}

  \end{center}
  \end{titlepage}
  
  % Ends the declared geometry for the titlepage
  \restoregeometry
}